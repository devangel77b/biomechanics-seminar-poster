\documentclass{article}

\usepackage{graphicx}
\usepackage[american]{isodate}
\dateinputformat{american}
\usepackage{fullpage}

\author{MIDN 1/C Sharat Nemani}
\title{Affordable Unmanned Aerial Systems (UAS), sensors, modular payloads, and algorithmic tools for ecological study}
\date{\printdate{2/28/2019}}


\begin{document}
\thispagestyle{empty}
\makeatletter
\vfill
\begin{center}
\includegraphics[width=2.5in]{RCE-logo-01.png}\\
\textbf{Biomechanics Seminar Series} is proud to present\\
\vspace{1em}
\textbf{\Large\@author}\\
\vspace{1em}
\textbf{\Large\@title}
\end{center}
\makeatother

\begin{abstract}
Ecological and behavioral studies and conservation efforts are often complicated by the need to gather data in remote or inaccessible areas. In military missions, unmanned aerial systems (UAS) have been instrumental in providing remote access and persistent presence. We will discuss an on-going design effort to develop affordable UAS designs, as well as sensors and modular payloads, aimed at supporting science missions.  Our designs leverage developments in hobby radio/control and autonomous flight, 3D printing, and rapid prototyping. We consider both fixed-wing and quadrotor designs and will discuss modular payloads, including visible and thermal imaging and automated image processing. The rugged nature of biological field studies and unique challenges help drive design innovations. We will present concept designs and initial prototyping for notional biology and ecology missions: (1) to image and count organisms and/or nests along a shore; (2) to survey an area, field, or stretch of river, repeatedly to identify areas of interest or to assess damage after natural disasters; and (3) to recover a sample from an inaccessible location on an island top or inside a sinkhole. We will also discuss the logistics of training undergraduate engineers as UAS operators (USNA's "School of Drones") and plans to deploy in support of biologists to accomplish science missions. The UAS designs we develop will be made available for science work, and we hope to connect with potential future missions using UAS to support biological studies.
\end{abstract}

\makeatletter
Please join us in \textbf{Maury Hall 111} on \textbf{\@date\ from 1250--1320}. \textbf{Light refreshments will be provided.}
\makeatother
\vfill

\begin{center}
\includegraphics[width=3in]{turtle1.jpg}
\end{center}
\vfill

\end{document}

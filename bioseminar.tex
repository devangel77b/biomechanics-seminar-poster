\documentclass{article}

\usepackage{graphicx}
\usepackage[american]{isodate}
\dateinputformat{american}
\usepackage{fullpage}

\author{MIDN 1/C Marc Descour}
\title{Soft robotic designs inspired by leeches}
\date{\printdate{2/21/2019}}


\begin{document}
\thispagestyle{empty}
\makeatletter
\vfill
\begin{center}
\includegraphics[width=2.5in]{RCE-logo-01.png}\\
\textbf{Biomechanics Seminar Series} is proud to present\\
\vspace{1em}
\textbf{\Large\@author}\\
\vspace{1em}
\textbf{\Large\@title}
\end{center}
\makeatother

\begin{abstract}
Soft robotic designs have the potential to provide improved maneuverability and durability compared to hard-bodied robots. We present on-going work to examine soft robotic designs inspired by leeches (Hirudinea: Lamarck, 1818). The segmented body plan of annelids is appealing for engineering designs in which pressure can be used as a means of controlling movement. We will discuss locomotion and attachment as observed in live leeches, then we will present new leech-inspired designs. By constructing a soft bodied robot with segmented soft pneumatic actuators and controlling them according to biologically inspired gaits, we hope to accomplish various modes of locomotion that could be useful within a bulk fluid, along a surface, within a thin film, or along an interface. The soft pneumatic actutors must control bending along the body, so we will discuss the relationship between input pressure, curvature, bend angle, and speed within a body segment. We will also discuss methods for attachment/detachment at the anterior and posterior ends of the robot. The locomotion and attachment designs will ultimately be combined in a soft robot capable of leech-like locomotion, which could be useful in search and rescue in challenging environments, in soft designs for medical devices, or in locomotion systems designed to traverse multiple environments in the presence of free surfaces or boundaries. Biomechanics and engineering are a two-way street; constructing bio-inspired soft bodied robots may provide further quantitative insight into the form and function of the organisms that served as the original inspiration. 

\end{abstract}

\makeatletter
Please join us in \textbf{Maury Hall 111} on \textbf{\@date\ from 1250--1320}. \textbf{Light refreshments will be provided.}
\makeatother
\vfill

\begin{center}
\includegraphics[width=3in]{1-leech_grande.jpg}
\end{center}
\vfill

\end{document}
